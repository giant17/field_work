\documentclass{article}

\usepackage[backend=biber, style=apa]{biblatex}
\usepackage{tikz}

\author{Gianluca Tadori}
\title{Field Work}

\begin{document}
\maketitle

\section{Morphology}

\subsection{Does this language have a \textbf{gender} distinction? How many genders are there?}

This language has a clear gender distinction and there are only two gender, male and female.

Pronoun did not differentiate between the gender, but a verb. When we use a verb, the end tell us about the gender.

wo likhta hai
wo likhti hai

he writes
she writes

main parta hoon
main parti hoon

I read, as he
I read, as she


wo kaam kar rahi hai
wo kaam kar raha hai

She is working
He is working

In this case, however, the gender is given by the helping verb

tum sub bhagou ge
tum sub bhagou gay

You all will run, for female
You all will run, for male

tum sub bhagete huo		You all run, female
tum sub bhagetay huo	You all run, male

\subsection{How are \textbf{plurals} formed in this language? List all the forms. Can you find phonetic factors}

There is a suffix at the end of the main verb, e ay male female

ap ghate hain second plural female (you sing)
ap ghatay hain second plural male (you sing)

tum sub ghou ge     you all will sing
tum sub ghou gay    you all will sing

tum sub ghate the   you all had sung
tum sub ghatay thay you all had sung


tum ghate the 		you sung - female
tum ghatay thay 	you sung - male

you sing, as female plural
you sing, as male plural

huo: present
ge/gay: future
the/thay: past
% TODO: phonetics factors

Are some forms used only wit hcerrtain genders?

Yes,

future: auxiliary
present: main verb
past: both




\subsection{Are there \textbf{inflectional prefixes} in this language? Are there circumfixes?}

Inflection prefix

na-khosh 	unhappy
khosh: happy

circumfixes



\end{document}


